\section{Related Work}
\label{sec:related}
We continue our discussion of evaluations from the Introduction, where we 
discussed programming language designs. We now
discuss evaluations of language implementations and of visualization
tools.

To our knowledge, the first empirical evaluations in the programming
languages fields were of programming language implementations
(e.g. compilers, and, later, Just-in-Time engines); 
SPEC was formed to curate nominally-impartial benchmark suites,
which would evaluate the performance of
computer systems and programming language implementations. 
The influence of this early focus continues
today, with Blackburn and 17 other prominent programming languages
researchers
authoring~\cite{blackburn16:_truth_whole_truth_nothin_but_truth}. That
work describes errors (``sins'') in empirical evaluation of programming
language \emph{implementations and systems}.  Specifically, it proposed a
framework that identifies two categories of sins: sins of reasoning,
leading to unsound claims; and sins of exposition, leading to poorly
specified claims and evaluations. Their framework (and 
checklist~\cite{berger19:_check_manif_empir_evaluat}) provides
practitioners with methodological techniques for evaluating the
integrity of their work or the work of others.  However, Blackburn et al's
recommendations are most appropriate for
performance-related evaluations of implementations. 
The present work focusses more on holistic
evalutaions of designs.

Moving farther afield from the field of programming languages, but
much closer to our present goals,
\citeN{merino18:_system_liter_review_softw_visual_evaluat} survey
published work about software visualization tools. As with
program transformation tools, software visualizatoin tools operate in
an open-ended space where it is almost impossible to make clear
quantitative judgments of work. They investigate evaluation strategies
for visualization tools, including survey-based, anecdotal evidence,
usage scenarios/case studies, experiments, and examples; and data
collection methods including questionnaires, think-aloud, interview,
video recording, sketch drawing, and others. They propose guidelines
for future work in that space, recommending the use of case studies
and experiments that can show effectiveness of novel tools.




