\begin{abstract}
We have found that evaluating novel tools in the programming languages and software engineering area is challenging. Tools should advance the state of the art by proposing new techniques and evaluating them. For many tools, validation needs to include evidence that they are useful to developers in practice. 

Peer review aims to evaluate both the intellectual contributions and the usefulness of novel tools, based on evidence provided by authors. The present work focusses on how authors can best provide evidence (to peer reviewers and to the community) about the usefulness of their program transformation tools, based on our past experience with tool development. We wrote this work to codify best practices that we aim to follow ourselves, but we hope that it will also help others who are also aiming to improve the state of the art in the area of programming tools.

We use our JTestParametrizer tool as a case study. It aims to automatically refactor clones in test suites. Our goal was to figure out whether JTestParametrizer was useful in its current state; if not, we would figure out potentially fruitful directions to continue development. First, we ran the tool on a convenience sample of 18 benchmarks. We carefully examined the performance of our tool on this benchmark set. Next, we developed questionnaires and used manual assessments and pull requests submitted to developers to solicit feedback on the quality of the tool. Like an industrial Minimum Viable Product or a lo-fi prototype, we sought feedback on potential features for JTestParametrizer before actually implementing them. We did this by manually applying the effect of different potential features that the tool could have on cases used for pull requests. By studying feedback from these manually modified pull requests, we determined the factors that the practitioners care about the most in the context of refactoring unit tests, allowing us to formulate and support hypotheses about suitable features to implement in JTestParametrizer next.
\end{abstract}
