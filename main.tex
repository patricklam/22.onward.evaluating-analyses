\documentclass[sigplan, review, anonymous, 10pt]{acmart}
%\acmConference[OOPSLA 2020]{Object-Oriented Programming Systems, Languages and Applications}{15--20 November, 2020}{Chicago, IL}
\acmConference[Onward! Papers 2022]{Onward! Papers}{8--9 December, 2022}{Auckland, NZ}
\setcopyright{acmcopyright}
\copyrightyear{2022}
\acmYear{2022}
\acmDOI{}
%\acmJournal{PACMPL}
%\acmVolume{OOPSLA}
%\acmArticle{}
%\acmMonth{11}

\usepackage{booktabs} % For formal tables
\usepackage{natbib}
\usepackage{listings}
\usepackage[justification=centering]{caption}

\citestyle{acmauthoryear}

\definecolor{dkgreen}{rgb}{0,0.6,0}
\definecolor{gray}{rgb}{0.5,0.5,0.5}
\definecolor{mauve}{rgb}{0.58,0,0.82}

\lstset{
  frame=tb,
  language=Java,
  aboveskip=3mm,
  belowskip=3mm,
  showstringspaces=false,
  captionpos=b,
  columns=flexible,
  basicstyle={\small\ttfamily},
  numbers=none,
  numberstyle=\tiny\color{gray},
  numberblanklines=false,
  keywordstyle=\color{blue},
  commentstyle=\color{dkgreen},
  stringstyle=\color{mauve},
  breaklines=true,
  breakatwhitespace=true,
  tabsize=2,
  escapeinside=??
}

\graphicspath{ {fig/} }

\usepackage{hyperref}
\newcommand{\RomanNumeralCaps}[1]{\MakeUppercase{\romannumeral #1}}
\usepackage{url}

\usepackage{tikz}
\usepackage{natbib}
\usetikzlibrary{shadows,arrows}
% Define the layers to draw the diagram
\pgfdeclarelayer{background}
\pgfdeclarelayer{foreground}
\pgfsetlayers{background,main,foreground}

% Define block styles  
\tikzstyle{material}=[draw, fill=gray!20, text width=25em, text centered, minimum height=3em]
\tikzstyle{action}=[draw, fill=white!20, text width=25em, text centered, rounded corners, minimum height=3em]
\tikzstyle{process} = [action, text width=25em, minimum width=10em, minimum height=3em]
\tikzstyle{result} = [material, text width=25em, minimum width=10em, minimum height=3em]
\tikzstyle{text1} = [above, text width=25em]

\tikzstyle{line} = [draw, thick, color=black, -latex']
 
% Define distances for bordering
\newcommand{\blockdist}{1.3}
\newcommand{\edgedist}{1.5}

\newcommand{\process}[2]{node (p#1) [process] {{\Large #2}}}
\newcommand{\result}[2]{node (p#1) [result] {{\Large #2}}}

% Draw background
\newcommand{\background}[5]{%
  \begin{pgfonlayer}{background}
    % Left-top corner of the background rectangle
    \path (#1.west |- #2.north)+(-1.5,1.0) node (a1) {};
    % Right-bottom corner of the background rectanle
    \path (#3.east |- #4.south)+(+1.5,-0.5) node (a2) {};
    % Draw the background
    \path[thick, fill=white!20,rounded corners, draw=black, dashed]
      (a1) rectangle (a2);
    \path (a1.east |- a1.south)+(4.2,-0.7) node (u1)[text1]
      {\LARGE\textbf{#5}};
  \end{pgfonlayer}}

% Copyright
%\setcopyright{none}
%\setcopyright{acmcopyright}
%\setcopyright{acmlicensed}
\setcopyright{rightsretained}
%\setcopyright{usgov}
%\setcopyright{usgovmixed}
%\setcopyright{cagov}
%\setcopyright{cagovmixed}


% DOI
\acmDOI{10.475/123_4}

% ISBN
\acmISBN{123-4567-24-567/08/06}

\acmArticle{4}
\acmPrice{15.00}

\begin{document}
\title{Evaluating Program Transformation Tools}


\begin{abstract}
Program transformation tools take a program as input and propose changes to that program; ideally, the changes are useful to the program's developers. Depending on the tool, changes may be semantics-preserving (e.g. refactoring) or not (e.g. repair). Evaluating novel program transformation tools is challenging; many dimensions of interest for these evaluations require subjective judgment. 

The present work focusses on how authors can best provide evidence (to peer reviewers and to the community) about the usefulness of their program transformation tools, based on our past experience with tool development. Peer review aims to evaluate both the intellectual contributions and the usefulness of novel tools, based on evidence provided by authors. We wrote this work to codify best practices that we aim to follow ourselves, but we hope that it will also help others who are also aiming to improve the state of the art in the area of programming tools.

We use our JTestParametrizer tool as a case study, although it is not the focus of this work. JTestParametrizer aims to automatically refactor clones in test suites. Our initial goal was to figure out whether JTestParametrizer was useful in its current state; if not, we would figure out potentially fruitful directions to continue development. We selected a benchmark suite, collected quantitative data about the performance of our tool on the suite, carefully examined its output, and sought feedback from manually curated pull requests.
%First, we ran the tool on a convenience sample of 18 benchmarks. We carefully examined the performance of our tool on this benchmark set. Next, we developed questionnaires and used manual assessments and pull requests submitted to developers to solicit feedback on the quality of the tool. Like an industrial Minimum Viable Product or a lo-fi prototype, we sought feedback on potential features for JTestParametrizer before actually implementing them. We did this by manually applying the effect of different potential features that the tool could have on cases used for pull requests. By studying feedback from these manually modified pull requests, we determined the factors that the practitioners care about the most in the context of refactoring unit tests, allowing us to formulate and support hypotheses about suitable features to implement in JTestParametrizer next.
% determined the factors that the practitioners care about the most in the context of refactoring unit tests, allowing us to formulate and support hypotheses about suitable features to implement in JTestParametrizer next.
\end{abstract}


% The code below should be generated by the tool at
% http://dl.acm.org/ccs.cfm
% Please copy and paste the code instead of the example below.
%
%\begin{CCSXML}
%<ccs2012>
% <concept>
%  <concept_id>10010520.10010553.10010562</concept_id>
%  <concept_desc>Computer systems organization~Embedded systems</concept_desc>
%  <concept_significance>500</concept_significance>
% </concept>
% <concept>
%  <concept_id>10010520.10010575.10010755</concept_id>
%  <concept_desc>Computer systems organization~Redundancy</concept_desc>
%  <concept_significance>300</concept_significance>
% </concept>
% <concept>
%  <concept_id>10010520.10010553.10010554</concept_id>
%  <concept_desc>Computer systems organization~Robotics</concept_desc>
%  <concept_significance>100</concept_significance>
% </concept>
% <concept>
%  <concept_id>10003033.10003083.10003095</concept_id>
%  <concept_desc>Networks~Network reliability</concept_desc>
%  <concept_significance>100</concept_significance>
% </concept>
%</ccs2012>
%\end{CCSXML}
%
%\ccsdesc[500]{Computer systems organization~Embedded systems}
%\ccsdesc[300]{Computer systems organization~Redundancy}
%\ccsdesc{Computer systems organization~Robotics}
%\ccsdesc[100]{Networks~Network reliability}


\keywords{Refactoring, test cases, program analysis}

\maketitle

\section{Introduction}

\section{Related Work}
\label{sec:related}
We continue our discussion of evaluations from the Introduction, where we 
discussed programming language designs. We now
discuss evaluations of language implementations and of visualization
tools.

To our knowledge, the first empirical evaluations in the programming
languages fields were of programming language implementations
(e.g. compilers, and, later, Just-in-Time engines); 
SPEC was formed to curate nominally-impartial benchmark suites,
which would evaluate the performance of
computer systems and programming language implementations. 
The influence of this early focus continues
today, with Blackburn and 17 other prominent programming languages
researchers
authoring~\cite{blackburn16:_truth_whole_truth_nothin_but_truth}. That
work describes errors (``sins'') in empirical evaluation of programming
language \emph{implementations and systems}.  Specifically, it proposed a
framework that identifies two categories of sins: sins of reasoning,
leading to unsound claims; and sins of exposition, leading to poorly
specified claims and evaluations. Their framework (and 
checklist~\cite{berger19:_check_manif_empir_evaluat}) provides
practitioners with methodological techniques for evaluating the
integrity of their work or the work of others.  However, Blackburn et al's
recommendations are most appropriate for
performance-related evaluations of implementations. 
The present work focusses more on holistic
evalutaions of designs.

Moving farther afield from the field of programming languages, but
much closer to our present goals,
\citeN{merino18:_system_liter_review_softw_visual_evaluat} survey
published work about software visualization tools. As with
program transformation tools, software visualizatoin tools operate in
an open-ended space where it is almost impossible to make clear
quantitative judgments of work. They investigate evaluation strategies
for visualization tools, including survey-based, anecdotal evidence,
usage scenarios/case studies, experiments, and examples; and data
collection methods including questionnaires, think-aloud, interview,
video recording, sketch drawing, and others. They propose guidelines
for future work in that space, recommending the use of case studies
and experiments that can show effectiveness of novel tools.





\section{Reproducibility}
\label{sec:reproducibility}

A discussion of evaluation presupposes reproducible runs of the
underlying artifact. Science can only work if results are
reproducible.  The challenges in performing reproducible scientific
research range from technical issues concerning input data,
under-specification in methodology or metrics, obfuscated or
unavailable codebases, and selective or exaggerated
reporting~\cite{allison2018reproducibility}. As a field, computer
science has historically not valued reproducibility; \citeN{bajpai2017challenges}
discuss some reasons why, and \citeN{collberg16:_repeat_comput_system_resear}
attempted to empirically measure repeatability.

\citeN{krishnamurthi:_about_artif_evaluat} advocated for artifact evaluation in software research: ``If a paper makes, or implies, claims that require software, those claims must be backed up.''. Programming languages conferences now have artifact evaluation processes and are continuing to refine their processes. Per Krishnamurthi, ``Artifact evaluation encourages authors to produce reasonable artifacts, which are the cornerstone of future research.''.

We have learned that being able to re-run even one's own artifacts years after 
initial development is quite valuable; in a typical research situation,
students come and go, and the institutional knowledge about how to run a
tool may no longer be present, making it hard to reproduce and in particular to build
on one's past results. In this section, we briefly discuss challenges
to reproducibility and some of the ways we have overcome them.

Artifacts run in environments, which include hardware, OS kernels,
distributions, (versioned) packages, configurations, and file system
contents. Differences in the environment hinder redeployment of artifacts,
despite the age-old ``Write Once, Run Anywhere'' dream that was encapsulated in
Sun Microsystems's 1995 slogan.

An early pragmatic attempt\footnote{That is, by accepting the world as-it-is rather than by creating a self-contained insular environment.} to address this problem was \citeN{guo2011cde}'s CDE 
packaging framework, which recreates the file system tree of the source environment and ships the code along with its input data to an arbitrary environment by creating a fake root containing the mentioned artifacts in the destination machine. While Guo's idea of shipping the entire environment as an apparatus for reproducible software runs is interesting, that work only encapsulated the file system. rr\footnote{\url{https://rr-project.org/}} goes further in a particular direction and provides the ability to rerun a specific execution of a binary; however, it does not provide insight into the system as a whole, nor does it facilitate experimentation with related executions of the binary. To be fair, rr's anticipated use case was not research reproducibility.

CDE's packaging strategy has evolved into modern containerization technologies such as Docker and LXC, which use Linux kernel features to replicate an environment as accurately as possible, in isolation from other containers and the host operating system. Thus, containers provide a comprehensive framework for creating the desired environment in machines with different OS distributions and installed packages. 

This has led many researchers to advocate for containers as a solution for reproducible research. For instance, \citeN{boettiger2015introduction} proposed Docker for reproducible research, defining best practices in doing so, e.g., using the containers during development, test cases, and checks, and using Dockerfiles instead of manually typing in commands. Virtualization technologies, in general, can play an essential role in future research by providing a snapshot of the testing environment and the evaluation results. They are no panacea, though, and performance results are particularly ill-suited to reproduction under containers.

Without such arrangements, reproducing studies on programming languages and software evaluations can often be challenging. For instance, \citeN{Berger_Reproducing} follow the footsteps of \citeN{ray2014large} along with their GitHub repository, documenting the results and cross-checking the claims. In this effort, reproducing some of the case studies has proven complex and not been completely successful.

\paragraph{JParametrizer reproducibility case study.}
We make our discussion concrete by discussing our experience with JTestParametrizer. The context was that a previous student had graduated, leaving us the artifact itself, along with a master's thesis documenting the work. 

Fortunately, we were able to obtain the exact versions of the earlier 5 benchmarks, as the evaluated git commit hash of each of the benchmarks was included in the artifact. We also had access to the artifact itself. However, we found that we were missing an intermediate Excel file, which contained a set of potential refactoring nominees. This file was generated by the Deckard tool; unfortunately, the previous student did not document the 3 Deckard parameters used to generate the Excel file. Rerunning JTestParametrizer with a newly-generated Excel file seemed to produce slightly different results from those documented, as the new run produced some compilation errors, while the thesis reported no compilation errors.

In addition to the Deckard dependency, our tool also had a dependency on the Eclipse IDE. So, we relied on the version being available to us now having the same behaviour as an earlier version, and we needed the run configurations to be comparable.

To aid future reproducibility of our work, we created a Figshare article documenting versions and configurations of our tool's dependencies. However, this is still a manual process and subject to human error; in the best case, a second person would attempt to follow the directions, re-run the tool, and compare the results.

In the remainder of this work, we set aside the important question of reproducibility, and focus on how to evaluate the designs and artifacts in question.

\section{How to Evaluate}
As we've discussed, evaluating the usefulness of many program
transformation tools requires some subjective judgment. In this
section, we outline possible ways to evaluate such tools.

The designer of the tool will have a usage scenario for the 
transformations produced by the tool. This usage scenario is 
necessary but not sufficient to establish that the tool is useful.
At this level of abstraction, the usage scenario is simply something
that a reader considers so that they can decide, based on their experience, 
whether the tool seems like it would be useful to them or not. 
It is still too abstract to make concrete decisions about.

Typically, a program transformation tool generates a large set of
candidate transformations. Depending on the tool, some of these
transformations may be inappropriate. For instance, a refactoring
tool should only propose semantics-preserving transformations; on the
other hand, we would expect a program repair tool to change the semantics
of the program being repaired. 

In any case, even after ruling out 
inappropriate transformations, the tool would still have a large
number of candidates to consider, and must select some of the candidates to show
to the user. This selection process is a key factor in the usability of the tool.
% cite the tricoder paper
Promising tools may fail to be useful in practice because the selection is not
tuned to users' needs.

A key question is: which humans should evaluate the results of the tool?
Possible answers include the tool developers, arbitrary developers, or the developers
of particular software projects.

Along with ``which humans'', another key question is how to solicit
information from the humans. Questionnaires and user studies are two
ways to ask solicit information. One can ask for abstract opinions
(``would this tool be useful?'') or concrete opinions (``is this
specific proposed transformation useful?'') At the limit, perhaps the
best way to show that a tool in this space is useful in practice is by
submitting merge requests to maintainers of open-source projects---a 
highly specific and idiosyncratic user study which may be hard to generalize.

We will discuss best practices for this type of evaluation.

\section{Setting}
Researchers and practitioners have developed tools to carry out
automated and semi-automated program transformatoins.  Examples
include tools for bug finding, program repair, and even to some extent
program synthesis. These tools work in a large space of possibilities:
for instance, programs have many potential bugs, only some of which
matter.
% XXX citations
In more general terms, programming languages research aims to develop
techniques to reason about and manipulate programs. Evaluating
techniques for soundness can be tricky, but is in principle
objective. Soundness, however, is insufficient for a contribution to
be strong. One can additionally evaluate these techniques in terms of
power, or beauty. But often, the usefulness of the tool or technique
is a key factor in its value, which can be shown
experimentally. Techniques may be useful to other researchers, or to
end users.

The goal of this paper is to discuss best practices for evaluating
program transformation tools. When possible, we use our
JTestParametrizer tool as a case study. That tool proposes
refactorings based on static analysis results; we describe some of the
steps that we have taken evaluate the usefulness of our tool.

Evaluating tools in this space is often subjective---there is no
single right answer. Some tools, such as program refactoring tools,
are designed to only affect nonfunctional properties of the
software---particularly maintainability---yet different users have
different standards for maintainability.  Meanwhile, program repair
tools attempt to make the software (closer to) correct. Even in that
space, per Hovemeyer and Pugh, ``finding bugs is
easy''~\cite{hovemeyer04:_findin_bugs_easy}. Deciding which bugs are
important to fix (and don't cause undesired regressions) is hard and
requires a judgment call. Program synthesis is also an
underconstrained problem and humans must decide which one of many
potential solutions is most appropriate.

A key question, then, is: which humans should evaluate the results of the tool?
Possible answers include the tool developers, arbitrary developers, or the developers
of particular software projects.

One way to show that a tool is useful in practice is by submitting merge requests to maintainers of open-source projects. We will discuss best practices for this type of evaluation.


\section{Concrete Examples of Evaluations in PL Research}

Linking user bug reports and code changes for fixing those bugs are missing for several software projects since the bug tracking and version control systems are often maintained independently. There have been some solutions, such as ReLink~\cite{relink}, proposed for this problem. However, Bissyandé et al.~\cite{ee_buglinking} believed that the presentation of the effectiveness of ReLink is subject to several issues, including a reliability issue with their ground truth datasets in addition to the extent of their measurement. They proposed a benchmark for evaluating this bug-linking solution, and they designed some research questions for quantitative and qualitative assessment of the effectiveness of this tool. Furthermore, they applied their benchmark to ReLink to determine the strengths and limitations of this tool.

The $i^*$ modeling framework is a modeling language used in the early system modeling phase to help understand the problem domain. This framework has been widely used in research and some industrial projects. However, Estrada et al.~\cite{ee_framework} concluded that no empirical evaluation existed to identify the strengths and weaknesses of this framework. They presented an empirical evaluation of the $i^*$ framework using industrial case studies. They conducted their work in collaboration with an industrial partner who was using object-oriented and model-driven approaches for their software development. Estrada et al.\ report lessons learned from this experience showing the strengths and weaknesses of this framework and, they believe that this evaluation could play a crucial role in guiding extensions of the $i^*$ framework.

% you should briefly define construct validity in a short phrase
Questionnaires are one method for soliciting feedback. However, a questionnaire does not guarantee quality results because it is difficult to find the right engaged target audience for a technical software tool questionnaire. Furthermore, feedback from the questionnaires tends to have a lower level of detail.
Laugwitz et al.~\cite{laugwitz2008construction} designed a user experience questionnaire to get feedback on six factors: Attractiveness, Perspicuity, Efficiency, Dependability, Stimulation, and Novelty. Their results indicated a satisfying level of reliability and construct validity.

Another method for getting feedback on the quality of a software tool and qualitative evaluation is manual self-assessment. The feedback received from this assessment can be as detailed as required. This method does not need external help, but it could be vulnerable to potential unconscious bias and wrong assumptions about what factors would be the most crucial for the overall quality of the tool.

One of the best techniques to obtain valuable feedback for a specific software tool would be to contact potential customers directly about the quality of the changes that our tool has made to their codebase. Submitting pull requests is a method that makes this technique possible. This allows us to determine which factors are the most crucial for the quality of our tool from the potential customers' point of view. However, using this method to get feedback requires spending more time.

When using the methods that require external help for getting feedback on the overall quality of changes (such as using questionnaires or submitting pull requests), an essential consideration is that ethical standards should be held whenever a method seeks external help. Violations of those ethical standards can cause irremediable consequences.

An example of these consequences occurred when the University of Minnesota got a university-wide ban by the Linux kernel. One of their systems-security researchers submitted pull requests to the Linux kernel for a hidden purpose that they did not state, which the Linux Foundation deemed very unethical. Developers were offended that the university had purposely wasted the reviewers' time. This resulted in a university-wide ban following an email from Linux Foundation fellow Greg Kroah-Hartman which stated: ``I suggest you find a different community to do experiments on,'' and ``You are not welcome here.''~\cite{minnesota_banned}

\section{Selecting Benchmarks}
\label{sec:selecting-benchmarks}
What we find depends greatly on where we look---an aphorism that also applies to benchmark selection when evaluating programming languages and software engineering research. 
It might seem that the amount of open-source code available on the Internet today is boundless. However, the second author has observed that their graduate students find it challenging to define suitable benchmark suites to evaluate their research. 
In this section, we discuss tips that we have found to be useful when choosing benchmarks, including filters that are useful to apply to benchmark selection.

\subsection{Existing Benchmark Collections}
Industrial consortia and researchers have created collections of benchmarks for various purposes. SPEC, the Standard Performance Evaluation Corporation, contributed an early Java benchmark collection, SPEC JVM 98~\cite{SPECjvm98} and updated it in 2008~\cite{SPECjvm2008}. SPEC JVM was designed to measure the performance of both Java virtual machine client platforms and hardware systems. Later, researchers introduced the DaCapo benchmark collection~\cite{DaCapo_inproceedings}, consisting of open-source, real-world Java applications. 
They introduced new value, time-series, and statistical metrics for static and dynamic properties such as code complexity, code size, heap composition, and pointer mutations. These benchmarks aimed to empirically evaluate performance and memory management efficiency of Java virtual machines.

From a more software engineering point of view, the Qualitas Corpus~\cite{QualitasCorpus} is, per the authors, ``[\ldots] a curated collection of software systems intended for analyzing empirical studies of code artifacts with the primary goal of providing a resource that supports reproducible studies of software.''. However, Qualitas's open-source Java software systems are not necessarily executable. Thus, XCorpus~\cite{XCorpus} is ``a set of 76 executable, real-world Java programs, including a subset of Qualitas Corpus''. To support execution, ``[XCorpus] uses a harness that is a combination of built-in and generated test cases, resulting in a branch coverage that is significantly better than what is available from DaCapo.''.

\subsection{Filters}
Different benchmarks are going to be more or less suitable for exploring different aspects of tools. We acknowledge that being too selective with benchmarks can amount to unfair cherry-picking. However, we claim that applying principled and defensible filters to a large initial benchmark set can be a reasonable way to proceed. Indeed, one might legitimately apply different filters for different intended uses of the benchmarks: if simply collecting quantitative data, one could cast the net broadly and include almost all of the benchmarks. (For instance, when evaluating compiler performance and aiming to produce valid claims, one should usually not cherry-pick benchmarks.) On the other hand, for labour-intensive case studies where one is hoping to submit pull requests, one would want to filter aggressively---it would be desirable to filter as objectively as possible, but being inclusive is not as important. Documenting the filtering criteria would be a best practice, though.

We discuss three filters that are useful to consider when soliciting certain types of feedback. Even if a project fails a filter, it might be good for certain types of evaluation. For instance, benchmark Joda-Time is in maintenance mode, but would still be suitable for developing potential refactorings to be evaluated by us, or by third parties through a questionnaire.

\paragraph{Maintenance Mode}
When proposing code improvements to the developers, one should check that a candidate benchmark is not in maintenance mode. For instance, after submitting the pull request for our Gson benchmark, we received the following feedback: ``This project is in maintenance mode, and we are generally going to be reluctant to accept PRs that are essentially cosmetic, especially if it is not trivially obvious that they do not change anything.''. We also prepared a pull request for Joda-Time, but realized that the project is in maintenance mode before submitting it.

We used the following heuristics to determine whether a project is in maintenance mode.
\begin{enumerate}
  \item Examine the last few commits to the codebase and check whether they are adding new features.
  \item Examine the time gap between the last few commits to see how frequently the codebase is being updated.
  \item Investigate if the developers mention maintenance mode in, for example, recent pull request messages.
\end{enumerate}

For instance, for Joda-Time, the most recent commit was about four months before we looked; the last few commits were all about a release (of a maintenance version); and the pull request message pointed out that they are in maintenance mode.

\paragraph{Estimated Response Time}
Another useful filter is to select projects with recent track records of fast responses to pull requests. For instance, Gson developers provided feedback to our pull request in less than an hour, whereas Bootique developers did not react to our pull request even after a month.

To evaluate the track record, one can look at a project's list of open/closed pull requests. The last few closed pull requests and some open pull requests will provide a reasonable estimate about how long it takes for developers to come up with feedback. Furthermore, the number of contributors can be an indicator too (one might expect a larger contributor pool to be in a position to provide faster feedback).

\paragraph{Familiarity with Benchmark's Domain}
To be able to best evaluate proposed changes to a project, familiarity with the domain of the project can useful, especially when changes are not guaranteed to be semantics-preserving (e.g. when writing synthesis or repair tools). It is unethical to submit known-bad changes; we would argue that researchers should be making a good-faith effort to ensure that their changes are correct and improve the target codebase. Understanding the codebase helps with that.

In our case, we wanted to rate potential refactorings; understanding the context of the potential refactoring was helpful.

%--- edited to here

\subsection{Constraints}

The JTestParametrizer tool forced some conditions on the candidate benchmarks, and I added some more conditions to either simplify the proce
ss of running the benchmarks or increase the potential quality of the feedback. Due to these constraints and because I knew that the feedbac
k I was looking for was mainly on the quality of the refactorings and not quantitative metrics such as performance, I decided to create a ne
w specific set of benchmarks for this work.

Here, I will discuss the main constraints on benchmarks for this part. The JTestParametrizer tool required:
\begin{enumerate}
  \item That the benchmarks be Java projects because the JTestParametrizer tool is designed for the Java programming language.
  \item That the benchmarks run and pass all the test runs successfully on my machines (I used a macOS machine to run the JTestParametrizer 
tool and an Ubuntu machine for running Deckard~\cite{DECKARD}).
  \item That the benchmarks run successfully on Eclipse because the JTestParametrizer is an Eclipse-based tool.
\end{enumerate}

Furthermore, here are three constraints that I added to simplify the process of running the benchmarks or increase the potential value of th
e feedback on the quality of refactorings:
\begin{enumerate}
  \item I added a constraint to use a setup based on the Maven project management tool. Having this constraint enables having the same build
 setup process for all of the benchmarks, which will lead to simplicity and consistency of the process. However, it limits the benchmarks to
 Maven projects.
  \item I added another constraint to restrict the minimum number of stars and forks on GitHub for candidate benchmarks to have more relevan
t and widely used benchmarks. Stars and forks are indicators of how many people decide to use or work on a specific project, and even though
 they are not the perfect metrics, to some degree, they show the reliability and the quality of that project and its developers. Therefore, 
this constraint can enhance the potential quality of feedback from the developers. Nevertheless, it limits the benchmarks to project with a 
certain minimum number of stars and forks.
  \item I added another constraint to discard the benchmarks that took more than one hour to run all the test runs on the macOS machine. Eac
h benchmark will be run repeatedly, and using the projects that take longer to build has no particular advantage to make up for the unnecess
ary time it takes. Therefore, this constraint will help to simplify the process of running benchmarks by saving time. However, it adds a tim
e constraint on the possible candidate benchmarks.
\end{enumerate}

\subsection{Process of Choosing the Benchmarks}

To find candidate benchmarks that satisfy the stated constraints, I started with the benchmarks that Jun Zhao used in his thesis~\cite{ZhaoJ
un2018}. And then, I went through over a hundred open source projects to find the candidates that fit our requirements.


I used the suggested Java projects on GitHub lists by \href{https://medium.com/issuehunt/50-top-java-projects-on-github-adbfe9f67dbc}{IssueH
unt}, \href{https://awesomeopensource.com/projects/maven-plugin}{awesomeopensource}, and \href{https://www.overops.com/blog/the-hitchhikers-
guide-to-github-13-java-projects-you-should-try/}{Henn Idan} to find potential open-source Java project candidates, although I discarded the
 Github projects where their number of stars plus their number of forks on GitHub was less than 800.

I also considered all the open-source Java projects from the \href{https://github.com/google/?q=&type=&language=java&sort=stargazers}{Google
}, \href{https://github.com/spotify/?q=&type=&language=java&sort=stargazers}{Spotify}, \href{https://github.com/apache/?q=&type=&language=java&sort=stargazers}{Apache}, \href{https://github.com/airbnb/?q=&type=&language=java&sort=stargazers}{Airbnb}, and \href{https://github.com/Netflix?q=&type=&language=java&sort=stargazers}{Netflix} companies on GitHub that had the minimum number of stars $+$ forks as potential benchmark candidates.

\subsection{Benchmark Requirements Checklist}

Checklists are a helpful way for quickly evaluating things. They help to be more organized and to not skip any vital step in the process.  I used the following checklist to determine if a candidate benchmark has the minimum requirements needed to be in the benchmark collection for this work.


\begin{enumerate}
  \item Ensure that the candidate is an open-source Java project built by Maven.
  \item Ensure that the candidate contains tests.
  \item Ensure that I can build the candidate on my machines (macOS and Ubuntu machines) and that all the test runs pass successfully; if this is not the case, spend up to one hour fixing the issues. If I cannot fix it, discard this candidate.
  \item Ensure that I can build the candidate on Eclipse; if this is not the case, spend one hour fixing the issues. If I cannot fix it, discard this candidate.
  \item Create the cluster files for that project using Deckard, and then use the cluster files to create the XLS file of the potential nominees for refactoring. Now, discard this candidate if the created XLS file is empty, meaning that there are no nominees for refactoring.
\end{enumerate}

If a project satisfied all these conditions, then I could consider it as a potential benchmark. However, these were the minimum requirements, and some projects had these requirements but still failed to make the benchmark collection due to having a low number of refactoring nominees or not having any refactored cases after running the JTestPrametrizer tool.


\subsection{Deckard and Potential Nominees for Refactoring} \label{section:deckard_explanation}

One of the primary inputs of the JTestParametrizer tool for every benchmark is an XLS file that includes basic information (folder, class, package, method, start line, end line, and clone group size) about the potential nominees for refactoring in the benchmark. I created this XLS file by running a process on the cluster files that I receive from running Deckard's clone detection tool on that benchmark.

To use Deckard's clone detection ability, we need to set up specific parameters such as {\sc Min\_tokens} (``minimum number of tokens required for clones''), {\sc Stride} (``size of the sliding window''), and {\sc Similarity} described in Deckard: scalable and accurate tree-based detection of code clones~\cite{DECKARD}.

Jun Zhao did not document the parameters he used in his research. I deduce that he used 0 for {\sc Stride} (equivalent to no merging of vectors), but I could not deduce what he used for {\sc Min\_tokens} and {\sc Similarity}. Increasing the {\sc Min\_tokens} will decrease the number of potential nominees for refactoring since the nominees have to have at least that many tokens to be eligible. On the other hand, if the {\sc Min\_tokens} is too low, we will have nominees that can be refactored, but the refactoring would not be worth doing since the nominees were very trivial. Furthermore, {\sc Similarity} should be a number between 0.9 and 1. If it is too high, we will miss several nominees (false negatives), and if it is too low, we will have poor nominees (false positives).

After considering eight different sets of values, {\sc Min\_tokens}=50--100, {\sc Stride}=2--0, and {\sc Similarity}=0.95--0.90, for these three parameters for the three benchmarks Gson, Joda-time, and Jfreechart, and evaluating the differences between the final XLS files created by each set of variables, I decided to use 50 for the {\sc Min\_tokens}, 0 for {\sc Stride}, and 0.95 for {\sc Similarity} for every benchmark to keep the results consistent. These chosen parameters produced good results for the three benchmarks that I was using for evaluation.

With 50 for {\sc Min\_tokens}, we will not have many false positives (nominees that can be refactored, but it would be better if not, because they are already trivial), and 50 is not too high to miss good nominees because of the eligibility constraint. Furthermore, Jiang et al.~\cite{DECKARD} state that for clone detection, {\sc Similarity} could be a number between 0.9 and 1, and with 0.95, it is not too high to miss on several nominees, and it is not too low to have poor nominees. Also, for the three benchmarks discussed above, changing the Similarity from 0.9 to 0.95 did not considerably impact the final list of nominees.

\subsection{Our Benchmark Suite}

In this section, I will go over the projects that I used as benchmarks for this work. First, I will go over the numeric facts of each benchmark in table \ref{table:benchmarks}, then I will explain each of the benchmarks briefly.

\paragraph{Numeric Facts}

Table \ref{table:benchmarks} lists the numbers related to the essential aspects of the GitHub repository, including the number of stars, number of forks, number of GitHub issues, number of contributors, number of closed pull requests, and number of open pull requests, on GitHub, for each of the projects used as benchmarks. As for the number of lines of java code, I used SLOCCount\footnote{Source Lines of Code Count, \url{https://dwheeler.com/sloccount/}} to measure it for the latest version of each benchmark.

There are 14 repositories in this table but 18 benchmarks in total. This is because two of the benchmarks (Netty/Codec-http and Netty/Buffer) are subprojects of the same GitHub repository, Netty. Also, I used two different versions of each of the three repositories Gson, Bootique, and Joda-time, as separate benchmarks.


\begin{table}[h!]
\centering
\resizebox{\textwidth}{!}{
\begin{tabular}{l r r r r r r r} 
 Repository & Stars & Forks & Lines of Java & GitHub Issues & Contributors & Closed PRs & Open PRs \\ [0.5ex] 
 \hline\hline
 Gson & 20k & 3.9k & 25.6k & 503 & 114 & 358 & 151 \\ 
 Jimfs & 2k & 245 & 17.4k & 26 & 23 & 96 & 4 \\
 Bootique & 1.3k & 286 & 18.5k & 31 & 17 & 84 & 4 \\
 Joda-time & 4.7k & 888 & 86.5k & 23 & 77 & 160 & 3 \\
 Commons-lang & 2.2k & 1.3k & 85.1k & - & 161 & 682 & 107 \\ 
 Commons-io & 800 & 519 & 41.5k & - & 76 & 232 & 29 \\
 Commons-collections & 506 & 339 & 67.6k & - & 57 & 215 & 28 \\
 Jfreechart & 732 & 296 & 133.5k & 65 & 22 & 75 & 65 \\
 Netty & 27.4k & 13.5k & 312.1k & 451 & 531 & 5,963 & 43 \\ 
 Checkstyle & 6.2k & 8k & 286.5k & 642 & 290 & 6,530 & 41 \\
 Git-commit-id-maven-plugin & 1.3k & 253 & 3.7k & 23 & 73 & 241 & 3 \\
 Docker-maven-plugin & 2.6k & 556 & 2.5k & 10 & 38 & 160 & 11 \\
 Maven & 2.7k & 2k & 91.9k & - & 143 & 428 & 55 \\
 Mybatis-3 & 16.1k & 10.9k & 60.8k & 123 & 182 & 1,156 & 55 \\ [1ex] 
\end{tabular}}
\caption{Numeric Characteristics of Benchmarks}
\label{table:benchmarks}
\end{table}

Table \ref{table:refactoring_nominees} lists the necessary information that we retrieved for each benchmark before running the JTestParametrizer tool on that benchmark. This information includes the version of that benchmark, the number of Java code lines for that specific version using SLOCCount, the number of tests run, failures, errors, and skipped from building that benchmark and running all the tests on my macOS machine, and finally, the number of refactoring nominees that I got from running a process on the cluster files that I got from running the Deckard clone detection tool on that version of the benchmark.

\begin{table}[h!]
\centering
\resizebox{\textwidth}{!}{
\begin{tabular}{l c r r r r r r} 
 Repository & Version & Lines of Java & Tests run & Failures & Errors & Skipped & Nominees \\ [0.5ex] 
 \hline\hline
 Gson & f649e05 & 25193 & 1050 & 0 & 0 & 1 & 39 \\ 
 Gson & f319c1b & 25269 & 1063 & 0 & 0 & 1 & 42 \\ 
 Jimfs & 3c9d8ba & 17472 & 5834 & 0 & 0 & 0 & 45 \\
 Bootique & d0648eb & 18589 & 231 & 0 & 0 & 0 & 22 \\
 Bootique & 9939bc6 & 18591 & 228 & 0 & 0 & 0 & 23 \\
 Joda-time & 0ae5311 & 86138 & 4222 & 0 & 0 & 0 & 261 \\
 Joda-time & 27edfff & 86536 & 4238 & 0 & 0 & 0 & 260 \\
 Commons-lang & 425d808 & 77224 & 4068 & 0 & 0 & 5 & 154 \\ 
 Commons-io & e4ff4a5 & 40336 & 1852 & 0 & 0 & 6 & 32 \\
 Commons-collections & 7d8b979 & 67647 & 16923 & 0 & 0 & 4 & 47 \\
 Jfreechart & d03e68a & 132452 & 2176 & 0 & 0 & 0 & 124 \\
 Netty/Codec-http & e69107c & 41014 & 858 & 0 & 0 & 0 & 47 \\ 
 Netty/Buffer & e69107c & 33564 & 10458 & 0 & 0 & 1198 & 20 \\ 
 Checkstyle & 6cbc1dc & 255741 & 3528 & 0 & 0 & 0 & 130 \\
 Git-commit-id-maven-plugin & 4a1ac8f & 7238 & 214 & 0 & 0 & 1 & 4 \\
 Docker-maven-plugin & 84020ac & 2434 & 59 & 0 & 0 & 0 & 7 \\
 Maven/Maven-core & 3fabb63 & 38653 & 388 & 0 & 0 & 4 & 26 \\
 Mybatis-3 & 1d82865 & 60825 & 1675 & 0 & 0 & 14 & 26 \\ [1ex] 
\end{tabular}}
\caption{Benchmarks' Refactoring Nominees}
\label{table:refactoring_nominees}
\end{table}

\subsection{Benchmarks for Different Types of Feedback}

As we previously discussed in chapter \ref{chap:evaluating}, we were aiming for different types of feedback.

Besides the three necessary constraints and the three unnecessary ones discussed earlier, we learned some factors that led us to create new optional constraints throughout the work. These newly created constraints were not necessarily for all benchmarks and were based on the type of feedback we wanted for that benchmark.


A refactoring case might be correct, meaning that it does not cause any errors or failures, and the behavior of the refactored test case stays the same, but that refactoring case might still be poor due to decreasing the maintainability or readability of the code. In that sense, verifying the correctness of refactorings and validating their quality are two different topics.

I used all 18 benchmarks to get numerical metrics and statistics. Besides, I used Gson, Joda-time, Jfreechart, Commons-lang, Bootique, and Jimfs benchmarks to verify the JTestParametrizer tool's correctness and to find the errors and bugs of the JTestParametrizer tool. If an error occurred when running the JTestParametrizer tool on a benchmark, or if after running the JTestParametrizer on a benchmark that benchmark had some compile errors, or if after running the JTestParametrizer on a benchmark, some of the tests run of that benchmark resulted in failures or errors, then I documented those incidents, and in most cases, tried to find the source of that problem and fix the JTestParametrizer tool.

I used Gson and Joda-time to develop practical examples for the questionnaire to validate the quality of the refactoring instead of the correctness (all the examples were correct), which we ended up not pursuing due to the reasons explained in chapter \ref{chap:qualitative}.

We manually validated refactoring quality for the benchmarks Jimfs, Gson, Joda-time, and Bootique. We also sought developer feedback about test refactorings of these benchmarks using pull requests.

We did not use all the benchmarks the same way. There were some benchmarks such as Checkstyle that we only used for getting numerical metrics and quantitative evaluation. Whereas, there were benchmarks like Gson that we used not only for that purpose but also for coming up with practical examples for the questionnaire, validating the quality of the refactorings by ourselves, and validating the quality of the refactorings by their developers.

\section{Conclusion}
In this work, we presented a principled approach for thinking about
evaluations of program transformation tools. After identifying
a set of benchmarks and a set of evaluators, one can pose questions
to the evaluators about the usefulness of the tool's output on the benchmarks.
We presented a case study---our JTestParametrizer program transformation tool---and
supplemented it with other examples of evaluations from the literature.
We hope to spark a discussion in the community about how to
productively evaluate this kind of tool, which we believe to be an
important application of program analysis techniques.



\bibliographystyle{plainnat}
\bibliography{bibliography}

\end{document}
