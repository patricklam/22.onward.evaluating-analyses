\section{Our Benchmark Suite}
\label{sec:our-benchmarks}
We next discuss how we constructed a benchmark suite for evaluating JTestParametrizer, using what we explained in Section~\ref{sec:selecting-benchmarks}.

The primary universe of projects that we used were from GitHub lists of suggested open-source Java projects maintained by community members. Specifically, we used lists from \href{https://medium.com/issuehunt/50-top-java-projects-on-github-adbfe9f67dbc}{IssueHunt}, \href{https://awesomeopensource.com/projects/maven-plugin}{awesomeopensource}, and \href{https://www.overops.com/blog/the-hitchhikers-guide-to-github-13-java-projects-you-should-try/}{Henn Idan} to find potential open-source Java project candidates. 
We also considered all the open-source Java projects from the \href{https://github.com/google/?q=&type=&language=java&sort=stargazers}{Google}, \href{https://github.com/spotify/?q=&type=&language=java&sort=stargazers}{Spotify}, \href{https://github.com/apache/?q=&type=&language=java&sort=stargazers}{Apache}, \href{https://github.com/airbnb/?q=&type=&language=java&sort=stargazers}{Airbnb}, and \href{https://github.com/Netflix?q=&type=&language=java&sort=stargazers}{Netflix} companies on GitHub. We filtered out projects where the number of stars plus forks was less than 800. This left us with over 100 candidate projects.

We also added benchmarks from previous iterations of the work, as documented in a master's thesis, to ensure that our results remained comparable with previously-reported results.

\paragraph{Benchmark Requirements Checklist}
Checklists are a helpful way to quickly evaluating things (empirical evaluations, as proposed by~\citeN{berger19:_check_manif_empir_evaluat}, and as inspired by \cite{gawande09:_check_manif}). In particular, using a checklist is an effective way to not skip steps in a routine process. We developed and used the following checklist:

\begin{enumerate}
  \item Ensure that the candidate is an open-source Java project built with Maven.
  \item Ensure that the candidate contains tests.
  \item Ensure that the candidate builds on our machines and that all the test runs pass successfully; if this is not the case, spend up to one hour fixing the issues.
  \item Ensure that the candidate builds in Eclipse; if this is not the case, spend one hour fixing the issues.
  \item Process the candidate using the Deckard clone detection tool, and then use the cluster files to create an Excel file of potential nominees for refactoring. Check that the created XLS file is not empty (there is at least 1 nominee for refactoring).
\end{enumerate}

We considered the projects that satisfied all of these conditions for inclusion in our set of benchmarks. We additionally filtered according to the number of refactoring nominees reported and refactored.

\paragraph{The Benchmarks}
Table~\ref{table:benchmarks} lists the benchmarks that we selected,
along with their properties. This data provides an overview of project
size (e.g. lines of code\footnote{via SLOCCount, \url{https://dwheeler.com/sloccount/}}), popularity (stars and forks), and vitality (open and
closed pull requests). Note that the table contains 14 repositories and 18 benchmarks; 2 benchmarks are subprojects of the same GitHub repository, and we used two versions of three of the projects (Gson, Bootique, Joda-Time) as separate benchmarks.
We also collected information about the git commit ID (version); number of tests and test failures; and the number of refactoring nominees, but omit them here as not sufficiently related to the purpose of this work.

\begin{table*}[h!]
\centering
\resizebox{\textwidth}{!}{
\begin{tabular}{l r r r r r r r} 
 Repository & Stars & Forks & Lines of Java & GitHub Issues & Contributors & Closed PRs & Open PRs \\ [0.5ex] 
 \hline\hline
 Gson & 20k & 3.9k & 25.6k & 503 & 114 & 358 & 151 \\ 
 Jimfs & 2k & 245 & 17.4k & 26 & 23 & 96 & 4 \\
 Bootique & 1.3k & 286 & 18.5k & 31 & 17 & 84 & 4 \\
 Joda-time & 4.7k & 888 & 86.5k & 23 & 77 & 160 & 3 \\
 Commons-lang & 2.2k & 1.3k & 85.1k & - & 161 & 682 & 107 \\ 
 Commons-io & 800 & 519 & 41.5k & - & 76 & 232 & 29 \\
 Commons-collections & 506 & 339 & 67.6k & - & 57 & 215 & 28 \\
 Jfreechart & 732 & 296 & 133.5k & 65 & 22 & 75 & 65 \\
 Netty & 27.4k & 13.5k & 312.1k & 451 & 531 & 5,963 & 43 \\ 
 Checkstyle & 6.2k & 8k & 286.5k & 642 & 290 & 6,530 & 41 \\
 Git-commit-id-maven-plugin & 1.3k & 253 & 3.7k & 23 & 73 & 241 & 3 \\
 Docker-maven-plugin & 2.6k & 556 & 2.5k & 10 & 38 & 160 & 11 \\
 Maven & 2.7k & 2k & 91.9k & - & 143 & 428 & 55 \\
 Mybatis-3 & 16.1k & 10.9k & 60.8k & 123 & 182 & 1,156 & 55 \\ [1ex] 
\end{tabular}}
\caption{Numeric Characteristics of Benchmarks}
\label{table:benchmarks}
\end{table*}



%% Table \ref{table:refactoring_nominees} lists the necessary information that we retrieved for each benchmark before running the JTestParametrizer tool on that benchmark. This information includes the version of that benchmark, the number of Java code lines for that specific version using SLOCCount, the number of tests run, failures, errors, and skipped from building that benchmark and running all the tests on my macOS machine, and finally, the number of refactoring nominees that I got from running a process on the cluster files that I got from running the Deckard clone detection tool on that version of the benchmark.

%% \begin{table}[h!]
%% \centering
%% \resizebox{\textwidth}{!}{
%% \begin{tabular}{l c r r r r r r} 
%%  Repository & Version & Lines of Java & Tests run & Failures & Errors & Skipped & Nominees \\ [0.5ex] 
%%  \hline\hline
%%  Gson & f649e05 & 25193 & 1050 & 0 & 0 & 1 & 39 \\ 
%%  Gson & f319c1b & 25269 & 1063 & 0 & 0 & 1 & 42 \\ 
%%  Jimfs & 3c9d8ba & 17472 & 5834 & 0 & 0 & 0 & 45 \\
%%  Bootique & d0648eb & 18589 & 231 & 0 & 0 & 0 & 22 \\
%%  Bootique & 9939bc6 & 18591 & 228 & 0 & 0 & 0 & 23 \\
%%  Joda-time & 0ae5311 & 86138 & 4222 & 0 & 0 & 0 & 261 \\
%%  Joda-time & 27edfff & 86536 & 4238 & 0 & 0 & 0 & 260 \\
%%  Commons-lang & 425d808 & 77224 & 4068 & 0 & 0 & 5 & 154 \\ 
%%  Commons-io & e4ff4a5 & 40336 & 1852 & 0 & 0 & 6 & 32 \\
%%  Commons-collections & 7d8b979 & 67647 & 16923 & 0 & 0 & 4 & 47 \\
%%  Jfreechart & d03e68a & 132452 & 2176 & 0 & 0 & 0 & 124 \\
%%  Netty/Codec-http & e69107c & 41014 & 858 & 0 & 0 & 0 & 47 \\ 
%%  Netty/Buffer & e69107c & 33564 & 10458 & 0 & 0 & 1198 & 20 \\ 
%%  Checkstyle & 6cbc1dc & 255741 & 3528 & 0 & 0 & 0 & 130 \\
%%  Git-commit-id-maven-plugin & 4a1ac8f & 7238 & 214 & 0 & 0 & 1 & 4 \\
%%  Docker-maven-plugin & 84020ac & 2434 & 59 & 0 & 0 & 0 & 7 \\
%%  Maven/Maven-core & 3fabb63 & 38653 & 388 & 0 & 0 & 4 & 26 \\
%%  Mybatis-3 & 1d82865 & 60825 & 1675 & 0 & 0 & 14 & 26 \\ [1ex] 
%% \end{tabular}}
%% \caption{Benchmarks' Refactoring Nominees}
%% \label{table:refactoring_nominees}
%% \end{table}

\paragraph{Discussion of Benchmarks}
As discussed above, we used different benchmarks for different purposes. As our tool was a refactoring tool, it was supposed to be semantics-preserving. Beyond that, though, it was also supposed to improve code quality (maintainability or readability). These were two independent qualities that needed to be established.

We used all 18 benchmarks to get quantitative metrics. Then, we used a subset of 6 benchmarks (Gson, Joda-time, Jfreechart, Commons-lang, Bootique, and Jimfs) to manually check that JTestParametrizer was indeed semantics-preserving. If an error occurred when running JTestParametrizer on a benchmark, or if after running the JTestParametrizer on a benchmark, that benchmark had compilation or test failures, then we documented those incidents, and in most cases, tried to find the source of that problem and fix the JTestParametrizer tool.

Moving to qualitative evaluations, we used Gson and Joda-Time as sources of practical examples for a questionnaire to validate the quality of the refactoring (which we did not eventually distribute).
We manually rated refactoring quality for 5 benchmarks (Jimfs, Gson, Joda-time, and Bootique). We also sought developer feedback about test refactorings of these benchmarks using merge requests.
We believe that, in a large space such as this one, it is appropriate to select specific points in the space for detailed exploration.
