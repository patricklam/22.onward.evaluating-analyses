\section{Evaluating Languages and Tools}
\label{sec:how-to-evaluate}
As we've discussed, evaluating the usefulness of many program
transformation tools requires some subjective judgment. 
Language evaluations (and in particular the ones we mention here) are
quite different from tool evaluations---language designers make
the design decisions upfront, potentially modifying the language
as they work through case studies, while tool designers should have
in mind a corpus of programs to transform.

We formulate the evaluation question as follows:
\begin{center}
\emph{How} is the tool being evaluated, on \emph{which subjects},
and by~\emph{who}?
\end{center}
This question works for all tools, but we will focus on program
transformation tools. The goal of the tool developers is to choose
an evaluation that will show that the tool is useful on an important
set of subjects for an appropriate audience.

\subsection{Concrete Evaluation Workflow}
We now consider the entire workflow involved in choosing concrete
program transformation suggestions to present to a developer. To present
a program transformation tool, researchers would explain decisions made during
the workflow and justify why they made those decisions, showing how they fit
with their
desired evaluation criteria.

\paragraph{Filtering bad candidates.} Typically, a
program transformation tool generates a large set of candidate
transformations. Depending on the tool, some of these transformations
are inappropriate and need to be filtered out. For instance, a
refactoring tool should only propose semantics-preserving
transformations; on the other hand, we would expect a program repair
tool to change the semantics of the program being repaired.
From a program analysis point of view, we are filtering out false
positives that our approximation proposes as candidates.

\paragraph{Selecting good candidates.} In any case, even after ruling out 
inappropriate transformations, the tool would still typically have a large
set of legal candidates to consider, and must select a tractable subset of the
candidates to show to the user. This selection process is a key factor
in the usability of the tool, and is one of the key lessons reported
in the Tricorder
work~\cite{sadowski18:_lesson_build_static_analy_tools_googl}.
(Tricorder passes on developer feedback to analysis writers
and expects each analysis to return mostly-actionable results, i.e.
to have low false-positive rates).
In general, promising tools may fail to be useful in practice because the
selection is not tuned to users' needs.

\paragraph{Evaluating the results.} At this stage, a program
transformation tool would have generated a ranked set of results on a
benchmark. The usefulness of the tool is in large part determined
by how it produces this ranked set of results.

\subsection{How, Which, and Who?}
We now return to the evaluation question, starting with the ``how'' subquestion.

\paragraph{How to evaluate?} The designer of the tool will have a usage scenario for the
transformations produced by the tool. This usage scenario is necessary
but not sufficient to establish that the tool is useful. It is simply
a first step---at this level of abstraction, the usage scenario is
simply something that a reader considers so that they can decide,
based on their experience, whether the tool seems like it would be
useful to them or not.  It is still too abstract to make concrete
decisions about.

One can create more elaborate usage scenarios, potentially
with stakeholder involvement.
For instance, moving away from program transformation tools, the $i^*$ modeling framework is a modeling language used in the early
system modeling phase to help understand a problem domain. This
framework has been widely used in research and some industrial
projects. However, \citeN{estrada06:_empir_evaluat_framew_model_based} pointed out that no
empirical evaluation existed to identify $i^*$'s strengths and
weaknesses. They presented an empirical evaluation of the $i^*$
framework using industrial case studies, conducting their work in
collaboration with an industrial partner. 
For some tools, stakeholder buy-in may be all that is required
to justify development of the tool; it would also be wise to
evaluate actual stakeholder usage once the tool has been deployed.
%Estrada et al.~report
%lessons learned from this experience showing the strengths and
%weaknesses of $i^*$, which they hope will play a crucial role in
%guiding extensions of the $i^*$ framework.

In general, language design heavily relies on case studies.
For instance, the creators of the Gator graphics programming language~\cite{geisler20:_geomet_types_graph_progr} used 8 self-created case studies to advocate for the language; the case studies serve to showcase language features, both in terms of expressiveness and in terms of potential bugs avoided.

Tool designers can solicit opinions along a continuum from abstract
opinions to concrete judgments. There is a continuum between the
abstract opinion ``would tool T be useful,'' through to ``do you
support this usage scenario X'', all the way to the concrete ``is
specific proposed transformation Y useful?''

Questionnaires, interviews, and user studies are some ways to solicit
information. In both cases, it is difficult to find an appropriate
target audience; ideally, the audience would provide engaged feedback
on the tool, but that is not always obvious.  Questionnaires can be
open-ended or closed-ended. Closed-ended questionnaries can yield
easy-to-summarize but coarse information, while open-ended
questionnaires require more effort from participants to answer and
more effort from researchers to analyze. Moving beyond open-ended
questionnaires, interviews are another way of getting feedback
from an audience. Open-ended questionnaires and interviews can be 
analyzed using a grounded theory framework~\cite{glaser67:_discov_groun_theor}.
\citeN{lubin21:_how} carry out interviews, analyze them using grounded
theory, and carry out additional user experiments (user studies)
to validate their findings.
User studies involve more work to set up than questionnaires but can also provide concrete
information about how a tool is actually used, typically to accomplish set tasks.

\paragraph{Which benchmarks?} Benchmark selection is important,
and we will discuss it in Section~\ref{sec:selecting-benchmarks}.

Exploratory, 
or pilot, studies can be useful to establish that the
anticipated usage scenarios actually occur in the wild, though. For instance,
we have had the unfortunate experience\footnote{Citation omitted for double-blind review.} of designing a static analysis,
and then having a hard time finding code to which it was
applicable. \cite{beckman11:_empir_study_objec_protoc_wild}, on the
other hand, have conducted an empirical study of 15 benchmark programs
and 2 million lines of code, establishing that 7.2\% of types defined
object protocols, and 13\% of classes were clients of such types.
Such examples also serve as concrete examples of a tool's usefulness.


\paragraph{Who evaluates?}
A key question is then: which humans should evaluate the
results of the tool?  Possible answers include the researchers
developing the tool; arbitrary developers; or the developers of
particular software projects.

Manual self-assessment of results by the tool developers is at least
useful in an exploratory phase. This method does not need external
help, and the tool developers can do as detailed an examination of
their results as they desire. However, it could be vulnerable to
potential unconscious bias and incorrect assumptions by tool
developers about what factors would be important to actual
users. Still, it is a necessary first step in tool evaluation, before
presenting the tool to an external audience.

Potential or actual tool users are the gold standard audience for
obtaining valuable feedback. Submitting merge requests to open source
projects is one way to ask for real feedback. Their acceptance (or
not) reveal a ground truth result about whether a project developer
considers the tool useful (though on a case that was hand curated by
the tool developers). Nevertheless, they provide some information
about which factors are the most crucial for the quality of a tool
from the potential customers' point of view. Doing a good job here
does take time.

At the limit, perhaps the
best way to show that a tool in this space is useful in practice is by
submitting merge requests to maintainers of open-source projects---a 
highly specific and idiosyncratic user study which may be hard to generalize.

When getting external feedback about a tool (e.g. questionnaires or
submitting merge requests), tool developers must consider ethics
(including Institutional Review Board approval when required). Violations of ethical
standards can cause severe consequences. For instance, the
University of Minnesota got a university-wide ban for contributing to
the Linux kernel. One of their systems security researchers submitted
merge requests to the Linux kernel for a hidden purpose that they did
not state, which the Linux Foundation deemed
unethical. Linux developers were offended that the university had purposely
wasted the reviewers' time. This resulted in a university-wide ban
following an email from Linux Foundation fellow Greg Kroah-Hartman
which stated: ``I suggest you find a different community to do
experiments on,'' and ``You are not welcome
here.''~\cite{minnesota_banned}





%We will discuss best practices for this type of evaluation.


%% ``An unsound claim can misdirect a field, encouraging the pursuit of unworthy ideas and the abandonment of promising ideas.''. S. M. Blackburn et al.~\cite{truth_Blackburn} believed that having a methodical approach to exploring, exposing, and addressing the root of unsound claims and poor exposition would help to solve this issue. 

% more IR, when there is ground truth
%% Linking user bug reports and code changes for fixing those bugs are missing for several software projects since the bug tracking and version control systems are often maintained independently. There have been some solutions, such as ReLink~\cite{relink}, proposed for this problem. However, Bissyandé et al.~\cite{ee_buglinking} believed that the presentation of the effectiveness of ReLink is subject to several issues, including a reliability issue with their ground truth datasets in addition to the extent of their measurement. They proposed a benchmark for evaluating this bug-linking solution, and they designed some research questions for quantitative and qualitative assessment of the effectiveness of this tool. Furthermore, they applied their benchmark to ReLink to determine the strengths and limitations of this tool.
